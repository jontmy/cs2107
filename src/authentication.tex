\textbf{Authentication} is the process of assuring that a communicating entity or the origin of a piece of information is that which it claims to be.

Authenticity implies integrity.

\section{Passwords}
Passwords are secrets shared between two entities, but unlike secret keys, passwords are human-generated and memorizable.

\begin{defn}{bootstrapping}
    The establishment of a user's password with a server, stored in a file together with their identity.

    A user can choose a password and send it to the server, or, the server can choose a default password. A secure communication channel is required.
\end{defn}

\begin{defn}{authentication}
    Password verification can take place \textit{interactively} where the user is prompted for their password, or \textit{non-interactively} where the password is sent to the server in plaintext.

    Access is granted if an only if the password is correct.
\end{defn}

However, weak authentication methods are vulnerable to \textbf{replay attacks} where an attacker can sniff information from the channel and replay it to impersonate the user.

\begin{defn}{password reset}
    Systems typically link an account to a recovery email address, allowing the user to reset their password via a link containing a one-time-password (OTP).

    Ownership of the email address proves the entity is authentic, as this is 2FA.
\end{defn}

\subsection{Attacks}
Default password stealing or password interception (e.g. mail) can occur during bootstrapping.

Passwords can also be stolen by \textbf{shoulder-surfing}, \textbf{sniffing} over unecrypted networks, \textbf{keylogging}, \textbf{phishing} (social engineering, including spear-phishing), \textbf{cache-exploitation}, or \textbf{insider attacks}.

Regular training, phishing exercises, and blacklisting of known phishing sites can combat phishing attacks.

\begin{defn}{dictionary attacks}
    Dictionary attacks are a brute-force attack where an attacker uses a pre-selected collection of phrases to guess the password.

    During an \textbf{online dictionary attack}, the attacker must interact with the authentication server, which acts an oracle.

    Severs can combat this by limiting the number of attempts or introducing delays between attempts.

    In an \textbf{offline dictionary attack}, the attacker has to first obtain some information about the password (e.g. an encrypted database) before searching through it without interacting with the server (e.g. a password-protected PDF).
\end{defn}

\begin{defn}{side-channel attacks}
    Side channel attacks rely on information leaked by the system, into the environment which can be intercepted by an attacker and used to derive the password.

    For example, sound and timing of keyboard presses.
\end{defn}

\subsection{\textit{n}-Factor Authentication}
Examples of \textbf{factors}:
\begin{enumerate}
    \keyitem*{something you know}{password, PIN}
    \keyitem*{something you have}{hardware token, smartphone, ATM card}
    \keyitem*{who you are}{biometrics}
\end{enumerate}

\begin{defn}{one-time-password (OTP) tokens}
    Some form of hardware installed with a secret key shared with the server that is used to generate an OTP for a specific and limited event (e.g. login, transaction).
    
    The password generation process can be either \textbf{time-based} or \textbf{sequence-based} (e.g. a counter for button presses).
\end{defn}

Because OTPs are limited in their validity, they cannot be reused and are therefore not vulnerable to replay attacks.

When comparing between possible hardware, e.g. a physical token or a smartphone app, they will each have their own security trade-offs, i.e. there is no single best solution.

\section{Biometrics}
Biometrics are physical characteristics of a user, e.g. fingerprints.

Unlike passwords, they can't be changed, shared, or forgotten, but they introduce a probability of \textbf{false matches} and \textbf{false non-matches}.

The \textbf{false match rate} (FMR) is the probability that a false biometric capture is incorrectly accepted as belonging to a user.

The \textbf{false non-match rate} (FNMR) is the probability that a genuine biometric capture is incorrectly rejected as not belonging to a user.

FMR and FNMR are inversely related --- a lower threshold for accepting a biometric capture increases the FMR and decreases the FNMR, and vice versa.

\subsection{Attacks}
Biometric systems are vulnerable to \textbf{spoofing} where an attacker uses a fake biometric capture to impersonate a user.

To combat this, biometric systems can use \textbf{liveness detection} where the system checks that the user is present and not an image, recording, etc..